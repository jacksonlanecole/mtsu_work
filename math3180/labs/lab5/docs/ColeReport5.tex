\documentclass[letterpaper,10pt,titlepage]{report}
\usepackage[letterpaper,margin=0.7in]{geometry}
\usepackage{tikz} \tikzset{>=latex}
\usepackage{pgfplots,tikz}
\usetikzlibrary{decorations.markings,arrows, calc, angles, quotes, decorations.pathmorphing,patterns, decorations.pathreplacing}
\pgfplotsset{compat=newest}
\usepackage{enumitem}
\usepackage{mathtools}
\mathtoolsset{showonlyrefs}
\usepackage{amssymb}
\usepackage{gensymb}
\usepackage{empheq}
\newcommand*\widefbox[1]{\fbox{\hspace{2em}#1\hspace{2em}}}
\usepackage{pdfpages}
\pagenumbering{gobble}
\newcommand{\itemn}{\newpage \setcounter{equation}{0} \item}
\usepackage{bm}
\usepackage{mathtools}
\usepackage{amsmath}
\usepackage{esint}
\usepackage{dirtytalk}
\usepackage{mathcomp}
\usepackage{mdframed}
\usepackage[pdfpagelabels]{hyperref}
%\begin{mdframed}[backgroundcolor=gray!20]
%\vspace{6pt}
%\end{mdframed}
\usepackage[margin=0.5in]{caption} %Allows for use of \caption* for non-labeled captions
\usepackage{graphicx}
\newcommand{\Lagr}{\mathcal{L}}
\graphicspath{ {images/} }
\usepackage{fancyhdr}

\newcommand{\infinitesum}[2]{\sum_{#1 = #2}^{\infty}}
\newcommand{\goodprime}{^{\prime}}
\newcommand{\tsi}[1]{\int_{\phi=0}^{2\pi}\int_{\theta=0}^{\pi}\int_{r=0}^{#1}}
\newcommand{\functionof}[2]{#1\left(#2\right)}
\newcommand{\unit}[1]{\,\hat{\bm{#1}}}
\newcommand{\vect}[1]{\mathbf{#1}}
\newcommand{\paren}[1]{\left(#1\right)}
\newcommand{\brackets}[1]{\left[#1\right]}
\newcommand{\abs}[1]{\left|#1\right|}
\newcommand{\eval}[3]{\left.#1\right|_{#2}^{#3}}
\newcommand{\anglers}[1]{\langle #1 \rangle}
\newcommand{\sinp}[1]{\sin{\paren{#1}}}
\newcommand{\cosp}[1]{\cos{\paren{#1}}}
\newcommand{\tanp}[1]{\tan{\paren{#1}}}

\newcommand{\sinpR}[2]{\sin^{#2}{\paren{#1}}}
\newcommand{\cospR}[2]{\cos^{#2}{\paren{#1}}}
\newcommand{\tanpR}[2]{\tan^{#2}{\paren{#1}}}


\newcommand{\expp}[1]{\exp\paren{#1}}
\newcommand{\thefrac}{\dfrac{n\pi}{d}}
\newcommand{\assolegendre}[1]{P_{#1}\paren{\cos{\theta}}}

\newcommand{\partialwoa}[1]{\dfrac{\partial}{\partial #1}}
\newcommand{\partiald}[2]{\dfrac{\partial}{\partial #2}\brackets{#1}}
\newcommand{\partialdd}[2]{\dfrac{\partial^2}{\partial^2 #2}\brackets{#1}}
\newcommand{\derivative}[2]{\dfrac{d #1}{d #2}}
\newcommand{\integral}[2]{\int_{#1}^{#2}}

\newcommand{\sn}[2]{#1\times10^{#2}}

%\begin{mdframed}[backgroundcolor=gray!20]
%\vspace{6pt}
%\end{mdframed}

\newcommand{\homeworkTitle}{Lab 05}
\newcommand{\homeworkDueDate}{October 04, 2018}
\newcommand{\homeworkClass}{MATH 3180}
\newcommand{\homeworkClassInstructor}{Dr. Seo}
\newcommand{\homeworkAuthorName}{\textbf{Jackson Cole}}

\title{\homeworkTitle\\
\homeworkClass: Numerical Analysis}
\author{Jackson Cole}
\date{October 04, 2018}


\begin{document}
\maketitle
\tableofcontents
% The following makes equations look much nicer
\setlength{\abovedisplayskip}{10pt}
\setlength{\belowdisplayskip}{10pt}
\setlength{\abovedisplayshortskip}{10pt}
\setlength{\belowdisplayshortskip}{10pt}
\includepdf[pages=-]{Lab5_LinearSystems_Fall_2018.pdf}
\topmargin=-0.45in
\headsep=0.25in


\pagestyle{fancy}
\rhead{\homeworkAuthorName}
\chead{\homeworkClass\ (\homeworkClassInstructor): \homeworkTitle}
%% \rhead{\firstxmark}
%% \lfoot{\lastxmark}
%% \cfoot{\thepage}

\section{Description of Experiment}
In this experiment, our goal was to implement a Naive Gaussian Elimination
method as well as a method for Gaussian Elimination with Partial Pivoting in
C++. Both of these methods also require implementation of back substitution
given a general matrix. We were also asked to read in data from a data file of a
given format (this data is shown in the next section, but is read in from a data
file in practice). We were to implement these methods to work for all data and
for other data files as well.

\section{Program Input}
\begin{verbatim}
2
3     2     4
-6    -4    6
 
2
6     -3    6
-2    1     -2
 
2
0     2     4
1     -1    5
 
3
1     1     2     4
1     1     0     2    
0     1     1     0
 
3
3     2     -5    0
2     -3    1     0
1     4     -1    4
 
3
3     4     3     10
1     5     -1    7
6     3     7     15
 
4
6     -2    2     4     16
12    -8    6     10    26
3     -13   9     3     -19
-6    4     1     -18   -34
 
4
1     -1    2     1     1
3     2     1     4     1
5     8     6     3     1
4     2     5     3     -1
 
5
9     3     2     0     7     35
7     6     9     6     4     58
2     7     7     8     2     53
0     9     7     2     2     37
7     3     6     4     3     39
\end{verbatim}

\section{Program Output}
\begin{verbatim}
You may now enter the path to a data file or choose the
default by pressing enter. The default is './data.txt'.
>
********************************************************************************
Original Augmented Matrix:
   3.000   2.000   |   4.000
  -6.000  -4.000   |   6.000

Naive Gaussian Elimination
> Upper triangular matrix obtained:
   3.000   2.000   |   4.000
   0.000   0.000   |  14.000
> SOLUTION:
x1 = -inf
x2 = inf

Gaussian Elimination with Partial Pivoting
> Upper triangular matrix obtained:
   3.000   2.000   |   4.000
   0.000   0.000   |  14.000
> SOLUTION:
x1 = -inf
x2 = inf
********************************************************************************

********************************************************************************
Original Augmented Matrix:
   6.000  -3.000   |   6.000
  -2.000   1.000   |  -2.000

Naive Gaussian Elimination
> Upper triangular matrix obtained:
   6.000  -3.000   |   6.000
   0.000   0.000   |   0.000
> SOLUTION:
x1 = nan
x2 = nan

Gaussian Elimination with Partial Pivoting
> Upper triangular matrix obtained:
   6.000  -3.000   |   6.000
   0.000   0.000   |   0.000
> SOLUTION:
x1 = nan
x2 = nan
********************************************************************************

********************************************************************************
Original Augmented Matrix:
   0.000   2.000   |   4.000
   1.000  -1.000   |   5.000

Naive Gaussian Elimination
> Upper triangular matrix obtained:
   0.000   2.000   |   4.000
     nan    -inf   |    -inf
> SOLUTION:
x1 = nan
x2 = nan

Gaussian Elimination with Partial Pivoting
> Upper triangular matrix obtained:
   1.000  -1.000   |   5.000
   0.000   2.000   |   4.000
> SOLUTION:
x1 = 7.000
x2 = 2.000
********************************************************************************

********************************************************************************
Original Augmented Matrix:
   1.000   1.000   2.000   |   4.000
   1.000   1.000   0.000   |   2.000
   0.000   1.000   1.000   |   0.000

Naive Gaussian Elimination
> Upper triangular matrix obtained:
   1.000   1.000   2.000   |   4.000
   0.000   0.000  -2.000   |  -2.000
   0.000     nan     inf   |     inf
> SOLUTION:
x1 = nan
x2 = nan
x3 = nan

Gaussian Elimination with Partial Pivoting
> Upper triangular matrix obtained:
   1.000   1.000   0.000   |   2.000
   0.000   1.000   1.000   |   0.000
   0.000   0.000   2.000   |   2.000
> SOLUTION:
x1 = 3.000
x2 = -1.000
x3 = 1.000
********************************************************************************

********************************************************************************
Original Augmented Matrix:
   3.000   2.000  -5.000   |   0.000
   2.000  -3.000   1.000   |   0.000
   1.000   4.000  -1.000   |   4.000

Naive Gaussian Elimination
> Upper triangular matrix obtained:
   3.000   2.000  -5.000   |   0.000
   0.000  -4.333   4.333   |   0.000
   0.000   0.000   4.000   |   4.000
> SOLUTION:
x1 = 1.000
x2 = 1.000
x3 = 1.000

Gaussian Elimination with Partial Pivoting
> Upper triangular matrix obtained:
   2.000  -3.000   1.000   |   0.000
   0.000   6.500  -6.500   |   0.000
   0.000   0.000   4.000   |   4.000
> SOLUTION:
x1 = 1.000
x2 = 1.000
x3 = 1.000
********************************************************************************

********************************************************************************
Original Augmented Matrix:
   3.000   4.000   3.000   |  10.000
   1.000   5.000  -1.000   |   7.000
   6.000   3.000   7.000   |  15.000

Naive Gaussian Elimination
> Upper triangular matrix obtained:
   3.000   4.000   3.000   |  10.000
   0.000   3.667  -2.000   |   3.667
   0.000   0.000  -1.727   |   0.000
> SOLUTION:
x1 = 2.000
x2 = 1.000
x3 = -0.000

Gaussian Elimination with Partial Pivoting
> Upper triangular matrix obtained:
   6.000   3.000   7.000   |  15.000
   0.000   4.500  -2.167   |   4.500
   0.000   0.000   0.704   |   0.000
> SOLUTION:
x1 = 2.000
x2 = 1.000
x3 = 0.000
********************************************************************************

********************************************************************************
Original Augmented Matrix:
   6.000  -2.000   2.000   4.000   |  16.000
  12.000  -8.000   6.000  10.000   |  26.000
   3.000 -13.000   9.000   3.000   | -19.000
  -6.000   4.000   1.000 -18.000   | -34.000

Naive Gaussian Elimination
> Upper triangular matrix obtained:
   6.000  -2.000   2.000   4.000   |  16.000
   0.000  -4.000   2.000   2.000   |  -6.000
   0.000   0.000   2.000  -5.000   |  -9.000
   0.000   0.000   0.000  -3.000   |  -3.000
> SOLUTION:
x1 = 3.000
x2 = 1.000
x3 = -2.000
x4 = 1.000

Gaussian Elimination with Partial Pivoting
> Upper triangular matrix obtained:
   3.000 -13.000   9.000   3.000   | -19.000
   0.000  24.000 -16.000  -2.000   |  54.000
   0.000   0.000   4.333 -13.833   | -22.500
   0.000   0.000   0.000  -0.462   |  -0.462
> SOLUTION:
x1 = 3.000
x2 = 1.000
x3 = -2.000
x4 = 1.000
********************************************************************************

********************************************************************************
Original Augmented Matrix:
   1.000  -1.000   2.000   1.000   |   1.000
   3.000   2.000   1.000   4.000   |   1.000
   5.000   8.000   6.000   3.000   |   1.000
   4.000   2.000   5.000   3.000   |  -1.000

Naive Gaussian Elimination
> Upper triangular matrix obtained:
   1.000  -1.000   2.000   1.000   |   1.000
   0.000   5.000  -5.000   1.000   |  -2.000
   0.000   0.000   9.000  -4.600   |   1.200
   0.000   0.000   0.000  -0.667   |  -3.000
> SOLUTION:
x1 = -7.233
x2 = 1.133
x3 = 2.433
x4 = 4.500

Gaussian Elimination with Partial Pivoting
> Upper triangular matrix obtained:
   5.000   8.000   6.000   3.000   |   1.000
   0.000  -2.600   0.800   0.400   |   0.800
   0.000   0.000  -1.154  -0.077   |  -3.154
   0.000   0.000   0.000   2.000   |   9.000
> SOLUTION:
x1 = -7.233
x2 = 1.133
x3 = 2.433
x4 = 4.500
********************************************************************************

********************************************************************************
Original Augmented Matrix:
   9.000   3.000   2.000   0.000   7.000   |  35.000
   7.000   6.000   9.000   6.000   4.000   |  58.000
   2.000   7.000   7.000   8.000   2.000   |  53.000
   0.000   9.000   7.000   2.000   2.000   |  37.000
   7.000   3.000   6.000   4.000   3.000   |  39.000

Naive Gaussian Elimination
> Upper triangular matrix obtained:
   9.000   3.000   2.000   0.000   7.000   |  35.000
   0.000   3.667   7.444   6.000  -1.444   |  30.778
   0.000   0.000  -6.303  -2.364   2.939   |  -7.939
   0.000   0.000   0.000  -8.500   0.288   | -24.346
   0.000   0.000   0.000   0.000  -0.681   |  -2.724
> SOLUTION:
x1 = 0.000
x2 = 1.000
x3 = 2.000
x4 = 3.000
x5 = 4.000

Gaussian Elimination with Partial Pivoting
> Upper triangular matrix obtained:
   7.000   3.000   6.000   4.000   3.000   |  39.000
   0.000   9.000   7.000   2.000   2.000   |  37.000
   0.000   0.000   0.667   1.333   0.333   |   6.667
   0.000   0.000   0.000   5.143   5.857   |  38.857
   0.000   0.000   0.000   0.000  -5.574   | -22.296
> SOLUTION:
x1 = 0.000
x2 = 1.000
x3 = 2.000
x4 = 3.000
x5 = 4.000
********************************************************************************
\end{verbatim}

\section{Conclusion/Findings}
We see clearly from several of the earlier matrices in this experiment that the
Naive Gaussian Elimination method indeed fails as a result of division by 0. In
many cases, we recieve a \texttt{nan} value for several of our solutions. This
problem is remedied by employing the method of partial pivoting during Gaussian
Elimination with Partial Pivoting. In any case, the resulting solutions from
either method are obtained by simply doing back substitution, which in both
cases is an identical process.

Given the power of modern computing, it seems as though there would not be very
many cases where performing Naive Gaussian Elimination in practice would be
beneficial, as the Gaussian Elimination with Partial Pivoting provides better
results in every case.

\end{document}

