\documentclass[letterpaper,10pt,titlepage]{report}
\usepackage[letterpaper,margin=0.7in]{geometry}
\usepackage{tikz} \tikzset{>=latex}
\usepackage{pgfplots,tikz}
\usetikzlibrary{decorations.markings,arrows, calc, angles, quotes, decorations.pathmorphing,patterns, decorations.pathreplacing}
\pgfplotsset{compat=newest}
\usepackage{enumitem}
\usepackage{mathtools}
\mathtoolsset{showonlyrefs}
\usepackage{amssymb}
\usepackage{gensymb}
\usepackage{empheq}
\newcommand*\widefbox[1]{\fbox{\hspace{2em}#1\hspace{2em}}}
\usepackage{pdfpages}
\pagenumbering{gobble}
\newcommand{\itemn}{\newpage \setcounter{equation}{0} \item}
\usepackage{bm}
\usepackage{mathtools}
\usepackage{amsmath}
\usepackage{esint}
\usepackage{dirtytalk}
\usepackage{mathcomp}
\usepackage{mdframed}
\usepackage[pdfpagelabels]{hyperref}
%\begin{mdframed}[backgroundcolor=gray!20]
%\vspace{6pt}
%\end{mdframed}
\usepackage[margin=0.5in]{caption} %Allows for use of \caption* for non-labeled captions
\usepackage{graphicx}
\newcommand{\Lagr}{\mathcal{L}}
\graphicspath{ {images/} }
\usepackage{fancyhdr}

\newcommand{\infinitesum}[2]{\sum_{#1 = #2}^{\infty}}
\newcommand{\goodprime}{^{\prime}}
\newcommand{\tsi}[1]{\int_{\phi=0}^{2\pi}\int_{\theta=0}^{\pi}\int_{r=0}^{#1}}
\newcommand{\functionof}[2]{#1\left(#2\right)}
\newcommand{\unit}[1]{\,\hat{\bm{#1}}}
\newcommand{\vect}[1]{\mathbf{#1}}
\newcommand{\paren}[1]{\left(#1\right)}
\newcommand{\brackets}[1]{\left[#1\right]}
\newcommand{\abs}[1]{\left|#1\right|}
\newcommand{\eval}[3]{\left.#1\right|_{#2}^{#3}}
\newcommand{\anglers}[1]{\langle #1 \rangle}
\newcommand{\sinp}[1]{\sin{\paren{#1}}}
\newcommand{\cosp}[1]{\cos{\paren{#1}}}
\newcommand{\tanp}[1]{\tan{\paren{#1}}}

\newcommand{\sinpR}[2]{\sin^{#2}{\paren{#1}}}
\newcommand{\cospR}[2]{\cos^{#2}{\paren{#1}}}
\newcommand{\tanpR}[2]{\tan^{#2}{\paren{#1}}}


\newcommand{\expp}[1]{\exp\paren{#1}}
\newcommand{\thefrac}{\dfrac{n\pi}{d}}
\newcommand{\assolegendre}[1]{P_{#1}\paren{\cos{\theta}}}

\newcommand{\partialwoa}[1]{\dfrac{\partial}{\partial #1}}
\newcommand{\partiald}[2]{\dfrac{\partial}{\partial #2}\brackets{#1}}
\newcommand{\partialdd}[2]{\dfrac{\partial^2}{\partial^2 #2}\brackets{#1}}
\newcommand{\derivative}[2]{\dfrac{d #1}{d #2}}
\newcommand{\integral}[2]{\int_{#1}^{#2}}

\newcommand{\sn}[2]{#1\times10^{#2}}

%\begin{mdframed}[backgroundcolor=gray!20]
%\vspace{6pt}
%\end{mdframed}

\newcommand{\homeworkTitle}{Lab 07}
\newcommand{\homeworkDueDate}{October 18, 2018}
\newcommand{\homeworkClass}{MATH 3180}
\newcommand{\className}{Numerical Analysis}
\newcommand{\homeworkClassInstructor}{Dr. Seo}
\newcommand{\homeworkAuthorName}{Jackson Cole}

\title{\homeworkTitle\\
\homeworkClass: \className}
\author{\homeworkAuthorName}
\date{\homeworkDueDate}


\begin{document}
\maketitle
\tableofcontents
% The following makes equations look much nicer
\setlength{\abovedisplayskip}{10pt}
\setlength{\belowdisplayskip}{10pt}
\setlength{\abovedisplayshortskip}{10pt}
\setlength{\belowdisplayshortskip}{10pt}
\includepdf[pages=-]{content/intropages-Labs6_and_7.pdf}
\topmargin=-0.45in
\headsep=0.25in


\pagestyle{fancy}
\rhead{\homeworkAuthorName}
\chead{\homeworkClass\ (\homeworkClassInstructor): \homeworkTitle}

\section{Description of Experiment}
In this experiment, we were tasked with constructing the Newtonian form of the
interpolating polynomial for the function
\begin{equation}
    \functionof{f}{x} = \dfrac{1}{x^2 + 1}.
\end{equation}
We then proceed to evaluating our constructed interpolating polynomial at
several data points within the bounds of the given data points (interpolating).
The program input and output are given below, where the program input is assumed
to be of the form given (i.e. number of values, followed by space separated
values, followed by number of values, followed by space separated values).

\section{Program Input}
\begin{verbatim}
9
-8 -6 -4 -2 0 2 4 6 8
17
-8 -7 -6 -5 -4 -3 -2 -1 0 1 2 3 4 5 6 7 8
\end{verbatim}

\section{Program Output}
\begin{verbatim}
How many data points?  9
Enter x values separated by spaces: -8 -6 -4 -2 0 2 4 6 8
How many test values?  17
Enter all x values separated by spaces: -8 -7 -6 -5 -4 -3 -2 -1 0 1 2 3 4 5 6 7 8

----------------------------------------
Divided Differences
----------------------------------------
Iteration: 0
---------------
   0.0153846154
   0.0270270270
   0.0588235294
   0.2000000000
   1.0000000000
   0.2000000000
   0.0588235294
   0.0270270270
   0.0153846154

Iteration: 1
---------------
   0.0153846154
   0.0058212058
   0.0158982512
   0.0705882353
   0.4000000000
  -0.4000000000
  -0.0705882353
  -0.0158982512
  -0.0058212058

Iteration: 2
---------------
   0.0153846154
   0.0058212058
   0.0025192613
   0.0136724960
   0.0823529412
  -0.2000000000
   0.0823529412
   0.0136724960
   0.0025192613

Iteration: 3
---------------
   0.0153846154
   0.0058212058
   0.0025192613
   0.0018588724
   0.0114467409
  -0.0470588235
   0.0470588235
  -0.0114467409
  -0.0018588724

Iteration: 4
---------------
   0.0153846154
   0.0058212058
   0.0025192613
   0.0018588724
   0.0011984836
  -0.0073131955
   0.0117647059
  -0.0073131955
   0.0011984836

Iteration: 5
---------------
   0.0153846154
   0.0058212058
   0.0025192613
   0.0018588724
   0.0011984836
  -0.0008511679
   0.0019077901
  -0.0019077901
   0.0008511679

Iteration: 6
---------------
   0.0153846154
   0.0058212058
   0.0025192613
   0.0018588724
   0.0011984836
  -0.0008511679
   0.0002299132
  -0.0003179650
   0.0002299132

Iteration: 7
---------------
   0.0153846154
   0.0058212058
   0.0025192613
   0.0018588724
   0.0011984836
  -0.0008511679
   0.0002299132
  -0.0000391342
   0.0000391342

Iteration: 8
---------------
   0.0153846154
   0.0058212058
   0.0025192613
   0.0018588724
   0.0011984836
  -0.0008511679
   0.0002299132
  -0.0000391342
   0.0000048918

-----------------------------------------------------------------------------------
  i                   x                f(x)               P8(x)      |f(x) - P8(x)|
-----------------------------------------------------------------------------------
  0              -8.000        0.0153846154        0.0153846154        0.0000000000
  1              -7.000        0.0200000000       -1.3682034976        1.3882034976
  2              -6.000        0.0270270270        0.0270270270        0.0000000000
  3              -5.000        0.0384615385        0.4198361257        0.3813745873
  4              -4.000        0.0588235294        0.0588235294        0.0000000000
  5              -3.000        0.1000000000       -0.1288247524        0.2288247524
  6              -2.000        0.2000000000        0.2000000000        0.0000000000
  7              -1.000        0.5000000000        0.7426929192        0.2426929192
  8               0.000        1.0000000000        1.0000000000        0.0000000000
  9               1.000        0.5000000000        0.7426929192        0.2426929192
 10               2.000        0.2000000000        0.2000000000        0.0000000000
 11               3.000        0.1000000000       -0.1288247524        0.2288247524
 12               4.000        0.0588235294        0.0588235294        0.0000000000
 13               5.000        0.0384615385        0.4198361257        0.3813745873
 14               6.000        0.0270270270        0.0270270270        0.0000000000
 15               7.000        0.0200000000       -1.3682034976        1.3882034976
 16               8.000        0.0153846154        0.0153846154        0.0000000000
-----------------------------------------------------------------------------------
\end{verbatim}

\section{Conclusion/Findings}
In this experiment, this method provides essentially the exact same results as
the Maple implementation we did in Lab 06. We see that the error of our
interpolating polynomial is 0 when evaluated at values of $x$ from which the
polynomial was constructed, but is greater when actually interpolating values
using our constructed polynomial.

\end{document}
